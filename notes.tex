\section{Basic notions}
\begin{definition}
A \defn{category} $C$ consists of a collection \footnote{These can be bigger than sets. I don't know the technical definition} $ob(C)$ of objects, and for every two objects $a,b\in ob(C)$ a collection $C(a,b)$ of maps satisfying the following:
\begin{itemize}
    \item If $f\in C(a,b)$ and $g\in C(b,c)$ then there exists $g\circ f\in C(a,c)$.
    \item There is a unique identity map $id_C\in C(c,c)$ such that $f\circ id_c=id_c\circ f=f$ for every $f$.
\end{itemize}
If $ob(C)$ is a finite set then we say $C$ is \defn{small}. If $C(a,b)$ is a set for every $a,b\in ob(C)$ then we say $C$ is \defn{locally small}
\end{definition}

\begin{definition}
A (covariant) \defn{functor} $F:C\rightarrow D$ between two categories is a mapping taking every $c\in ob(C)$ to an object $Fc\in D$ and every $f\in C(a,b)$ to a map $F(f)\in D(Fa,Fb)$ such that $F(g\circ f)=F(g)\circ F(f)$.

A \defn{contravariant functor} $F$ is a functor $F:C^{op}\rightarrow D$ from the opposite category. In other words, $F$ reverses the direction of arrows in $C$.
\end{definition}

\begin{definition}
A \defn{natural transformation} $\eta:F\rightarrow G$ between two functors $F,G:C\Rightarrow D$ is, for every $x\in C$ a mapping $\eta_x:Fx\rightarrow Gx$ (called the component at $x$) such that the following diagram commutes.
% https://tikzcd.yichuanshen.de/#N4Igdg9gJgpgziAXAbVABwnAlgFyxMJZABgBpiBdUkANwEMAbAVxiRADEAKADwEoQAvqXSZc+QigCM5KrUYs2AER78hI7HgJEyk2fWatEHTgE9VwkBg3ii03dX0Kjys4NkwoAc3hFQAMwAnCABbJDIQHAgkACYHeUNjP3N-INDEaQioxABmOIM2AHFOJMELQJCw6kikDMcEgB16mBw6AH1uUpSKxFjMpFy5fKNG5raTNwEgA
\[\begin{tikzcd}
F(x) \arrow[d, "F(f)"] \arrow[r, "\eta_x"] & D(x) \arrow[d, "G(f)"] \\
F(y) \arrow[r, "\eta_y"]                   & D(y)                  
\end{tikzcd}\]

If each component $\eta_x$ is an isomorphism, then we sat $\eta$ is a \defn{natural isomorphism}.
\end{definition}

Lets give many examples.
\begin{example}[Categories]
The following are categories.
\begin{itemize}
    \item The category $\Set$ of sets and functions.
    \item The category $\Cat$ of small categories and functors.
    \item The category $\CAT$ of locally small categories and functors.
    \item The category $\catname{Ab}$ of abelian groups and group homomorphisms.
    \item The category $\catname{Grp}$ of groups and group homomorphisms.
    \item The category $\catname{Ring}$ of rings and ring homomorphisms.
    \item The category $\catname{R-Mod}$ of $R-$modules and module homomorphisms.
    \item The category $\catname{Top}$ of topological spaces and continuous maps.
    \item The category $\catname{Top^*}$ of pointed topological spaces and pointed maps.
    \item The category $\catname{hTop}$ of pointed topological spaces and classes of homotopy equivalences.
    \item For a category $C$, the opposite category $C^{op}$ where arrow directions are reversed.
    \item For a category $C$ and an object $x$, the \defn{slice category} or \defn{over category} $C/x$ whose elements are maps into $x$ and maps commutative diagrams
    % https://tikzcd.yichuanshen.de/#N4Igdg9gJgpgziAXAbVABwnAlgFyxMJZABgBpiBdUkANwEMAbAVxiRDpAF9T1Nd9CKAIzkqtRizYAPLjxAZseAkTJCx9Zq0QgARlzEwoAc3hFQAMwBOEALZIyIHBCRDuF63cQAmak5duQK1t7X2dvag1JbXN9TiA
\[\begin{tikzcd}
a \arrow[r] \arrow[d, "f"] & x \\
b \arrow[ru]               &  
\end{tikzcd}\]
    \item The similarly defined \defn{under category} $x/C$ of morphisms out of $x$ and maps that form commutative triangles.
    \item The trivial category with a single object and a single identity map.
    \item For every group $G$ the category $C_G$ with one object and one map for every $g\in G$ with composition defined by the group law.
    \item A \defn{poset} $P$ is a category with a unique map $f:a\rightarrow b$ between any two objects, interpreted as a relation $a\leq b$.
    \item A \defn{discrete category} is a category whose only maps are identity maps.
    \item For any two categories $C,D$, there is a category $Fun(C,D)$ of functors between them and natural transformations.
\end{itemize}
\end{example}

\begin{example}[Functors] The following are functors.
\begin{itemize}
    \item There is a large class of "forgetful" functors that forget extra structure. For example, the functors $U:\catname{Grp}\rightarrow \Set$, $U:\catname{Top}\rightarrow \Set$, $U:\catname{Ab}\rightarrow \catname{Grp}$, $U:\catname{Top^*}\rightarrow \catname{Top}$ which in each case forget the extra structure.
    \item There are also a large class of "free" functors, for example $F:\Set\rightarrow \catname{Ab}$ mapping a set to the free abelian group on it, $\delta:\Set\rightarrow \catname{Top}$ giving a set the discrete topology, $\tau:\Set\rightarrow \catname{Top}$ giving a set the chaotic topology, and the abelianisation functor $F:\Grp\rightarrow \Ab$ sending a group to the free abelian group on it.
\end{itemize}
\end{example}

\begin{definition}
Two functors $F:C\rightarrow D$ and $G:D\rightarrow C$ are said to be an $\defn{equivalence of categories}$ if there exist natural isomorphisms $FG\iso id_C$ and $GF\iso id_D$.
\end{definition}

\section{Abelian categories}
Loosely, an \defn{additive category} is one where morphisms can be added, and where we generally act like the category of abelian groups. 
\begin{definition}
An \defn{additive category} is a category $C$ such that 
\begin{itemize}
    \item $Hom(a,b)$ is an abelian group under $+$, where $\circ$ distributes over $+$:
    $$f\circ (g+h)=(f+g)\circ (f+h)$$
    $$(f+g)\circ h=(f\circ h)+(g\circ h)$$
    \item $C$ has a \defn{zero object}\footnote{A \defn{zero object} is an initial and final object.} $0$.
    \item $C$ has finite (co)products and these two coincide.\footnote{In fact, is enough to demand it has finite products - the rest will follow.}
\end{itemize}
\end{definition}
The abelian group structure will be different for each pair $a,b\in C$. In the case of $Ab$, the group structure on $Hom(a,b)$ is given by the group structure on $b$, but $b$ needn't have a group structure in general. As an example, if $J$ and $A$ are categories with $J$ small and $A$ abelian, then $Fun(J,A)$ is an abelian (in particular additive) category.

We want to extend this definition to encompass all categories of "abelian things". In particular, we want to be able to take kernels and cokernels of maps - this will allow us to do homological algebra. Notice that the (co)kernel has a universal property as respectively a pullback and a pushout.

% https://tikzcd.yichuanshen.de/#N4Igdg9gJgpgziAXAbVABwnAlgFyxMJZABgBpiBdUkANwEMAbAVxiRAGsYAnACgDMAlCAC+pdJlz5CKAIzkqtRizYBBEWJAZseAkTkyF9Zq0QgAQuvHapRMgepHlp4pc0Sd05ACZ5DpSZA1UStJXRQAZl9FYzYLYLdrMORI+2inEABjCE5eQVctUM8fVMcAl2EFGCgAc3giUD4uCABbJDIQHAgkGXjGlu7qTqQfNIC+Vz7WxEiOrsQvXqap9qHpxf7EABZBuYBWPxjTcfWp-dmkADYTpG3zxAB2a4edy4rhIA
\[\begin{tikzcd}
ker(f) \pullback
\arrow[r] \arrow[d] & A \arrow[d, "f"] & A \arrow[r, "f"] \arrow[d] & B \arrow[d] \\
0 \arrow[r]                & B                & 0 \arrow[r]                & coker(f)   
\end{tikzcd}\]

\begin{definition}
An \defn{abelian} category is an additive category with kernels and cokernels. In addition, we require that every monomorphism is the kernel of its cokernel and every epimorphism is the cokernel of its kernel.
\end{definition}
The additional requirement correspond to how we expect injective and surjective homomorphisms to act. If $f$ is injective, then $\pi:B\rightarrow B/im f$ has kernel $im(f)\iso A$. If $f$ is surjective, then $A/ker(f)\iso im(f)=B$. These are not an immediate property of the (co)kernels as (co)limits\footnote{Can I give an example?}. Kernels and cokernels allow us to define images as $im(f)=ker(coker(f))$. These have pleasing properties: $A\rightarrow B$ factors uniquely through $imf\rightarrow B$ and $A\rightarrow im(f)$ is an epimorphism.

\textbf{Cool fact alert!}
\begin{theorem}[Freyd-Mitchell Embedding Theorem]
Every abelian category s.t. $Hom(X,Y)$ is a set embeds fully faithfully into $Mod_R$ for some (possibly non-commutative) ring $R$.
\end{theorem}
\begin{proof}
\cite{Weibel}
\end{proof}
Which abelian categories correspond to non-commutative rings? Also, is $R$ unique?

\begin{definition}
A covariant functor $F:A\rightarrow B$ is right-exact if it preserves right-exact sequences. Similar for left-exact.
% https://tikzcd.yichuanshen.de/#N4Igdg9gJgpgziAXAbVABwnAlgFyxMJZABgBpiBdUkANwEMAbAVxiRAEEQBfU9TXfIRQBGclVqMWbdgHJuvEBmx4CRAExjq9Zq0QcZcnn2WCiAZk0SdbYvOMDVKMsPHapegGKcji-iqHIoi5akrogXoYKSg4BGsFW7uGykfb+5qTxbmG2XOIwUADm8ESgAGYAThAAtkhkIDgQSMI+FdVN1A1Iai2VNYga9Y2IZj1tiAAsHUMArKN901NIAGxzy4uIAOy5XEA
\[\begin{tikzcd}
A \arrow[r]  & A' \arrow[r]  & A'' \arrow[r]  & 0 \\
FA \arrow[r] & FA' \arrow[r] & FA'' \arrow[r] & 0
\end{tikzcd}\]
A contravariant functor is left-exact if it takes right-exact sequences to left-exact sequences. Similar for right-exact.
\end{definition}

\begin{lemma}
Limits commute with limits and right adjoints. In particular, in an abelian category,
because kernels are limits, both right adjoints and limits are left-exact.

Dually, colimits commute with colimits and left adjoints. In particular, because cokernels are
colimits, both left adjoints and colimits are right-exact.
\end{lemma}